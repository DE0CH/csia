% Options for packages loaded elsewhere
\PassOptionsToPackage{unicode}{hyperref}
\PassOptionsToPackage{hyphens}{url}
%
\documentclass[
]{article}
\usepackage{amsmath,amssymb}
\usepackage{lmodern}
\usepackage{ifxetex,ifluatex}
\ifnum 0\ifxetex 1\fi\ifluatex 1\fi=0 % if pdftex
  \usepackage[T1]{fontenc}
  \usepackage[utf8]{inputenc}
  \usepackage{textcomp} % provide euro and other symbols
\else % if luatex or xetex
  \usepackage{unicode-math}
  \defaultfontfeatures{Scale=MatchLowercase}
  \defaultfontfeatures[\rmfamily]{Ligatures=TeX,Scale=1}
\fi
% Use upquote if available, for straight quotes in verbatim environments
\IfFileExists{upquote.sty}{\usepackage{upquote}}{}
\IfFileExists{microtype.sty}{% use microtype if available
  \usepackage[]{microtype}
  \UseMicrotypeSet[protrusion]{basicmath} % disable protrusion for tt fonts
}{}
\makeatletter
\@ifundefined{KOMAClassName}{% if non-KOMA class
  \IfFileExists{parskip.sty}{%
    \usepackage{parskip}
  }{% else
    \setlength{\parindent}{0pt}
    \setlength{\parskip}{6pt plus 2pt minus 1pt}}
}{% if KOMA class
  \KOMAoptions{parskip=half}}
\makeatother
\usepackage{xcolor}
\IfFileExists{xurl.sty}{\usepackage{xurl}}{} % add URL line breaks if available
\IfFileExists{bookmark.sty}{\usepackage{bookmark}}{\usepackage{hyperref}}
\hypersetup{
  hidelinks,
  pdfcreator={LaTeX via pandoc}}
\urlstyle{same} % disable monospaced font for URLs
\usepackage{listings}
\newcommand{\passthrough}[1]{#1}
\lstset{defaultdialect=[5.3]Lua}
\lstset{defaultdialect=[x86masm]Assembler}
\setlength{\emergencystretch}{3em} % prevent overfull lines
\providecommand{\tightlist}{%
  \setlength{\itemsep}{0pt}\setlength{\parskip}{0pt}}
\setcounter{secnumdepth}{-\maxdimen} % remove section numbering
%! suppress = FileNotFound
\usepackage[a4paper, margin=1in]{geometry}
%\linespread{1.5}
\usepackage{amsmath}
\usepackage{xcolor}
\usepackage{newpxtext}
\usepackage[euler-digits]{eulervm}


\lstset{
  basicstyle=\ttfamily,
  columns=fullflexible,
  frame=single,
  breaklines=true,
  postbreak=\mbox{\textcolor{red}{$\hookrightarrow$}\space},
}
\ifluatex
  \usepackage{selnolig}  % disable illegal ligatures
\fi

\author{}
\date{}

\begin{document}

\hypertarget{description-of-scenario}{%
\section{Description of Scenario}\label{description-of-scenario}}

My client is the psychology teacher in my school. For the subject
Psychology, students often need to quote studies to support their claims
in their exams or research paper. The categorization for the paper are
very specific to IB as well, such as the nine designated topics, the
three different approaches. There is also a very specific set of code
studies that are considered valid for the examinations. Students can
benefit greatly from a search engine that can filter the studies for
them as they would not need to manually go through all the studies
themselves to find the relevant one. The current search engine such as
Google Scholar and Jstor are not suitable for the specific scenario as
they contain a very large dataset, many of which are not suitable for
the IB Psychology. They also do not have the ability to search for
specific IB Psychology topics and approaches. They also do not give the
teacher the control to guide students through tagging documents.

\hypertarget{rational}{%
\section{Rational}\label{rational}}

I judged that using a web app would be most suitable for the Scenario.
More specifically, using Django + Python + React + Semantic UI as the
development frameworks, SQL, AWS S3 as the database and storage, and
deployed on Heroku. An web app is chosen over an conventional app for
several reasons. First, the database will be updated periodically by the
teacher, so it would be convenient if all the students and teacher share
the same database in order to eliminate the duplication of files.
Second, a web app would load faster as it does not require download from
the user. Third, it works cross platform as it would only require a web
browser from the user to run well, which exist on almost all platforms
such as Mac, Windows, Ubuntu Desktop and smartphones and tablets. It
significantly widens the compatibility of the program without requiring
the programmer to make a program for every platform. While cross
platforms tools such as electron does exist, those are still unnecessary
for the purpose of the this project as the project would not need to be
run offline. In addition, the city of the client has well built internet
infrastructure, so a slow or unavailable internet is not taken into
account. Lastly, making this a web app significantly reduces the
maintenance cost as an update can be pushed out instantly for everyone
is there is a feature update.

The reason to choose Python and Django as the back end framework is
predominately due to the connivence. Python is chosen over other
programming language such as Java and C++ because of the faster
development speed in trade of the slower runtime speed (Barot). The
program does not need to have excellent performance as there would not
be too much traffic. A web development framework would be very
beneficial as that eliminates the need to write an web app from scratch.
Plain HTML, CSS and Javascript (i.e.~a static website) is not suitable
for this project as the database will be dynamically built by the
client. For the exact reason, Django is chosen over other Python web
development frameworks such as Flask because it has a very well built
interface for SQL databases that allow the user to interface with Python
instead of SQL syntax (Singh).

I choose React + Semantic UI as the front end development framework
because it speeds up the development time as well. Semantic UI is
particularly useful as it has a very large library of beautiful elements
and modules for developers to incorporate into their design. Semantic UI
also has tight integration with React, which is a JavaScript UI
framework. JSX is especially useful as it is compact and declarative.

\hypertarget{success-criteria}{%
\section{Success Criteria}\label{success-criteria}}

\hypertarget{functionality}{%
\subsection{Functionality}\label{functionality}}

\begin{itemize}
\tightlist
\item
  Teacher can upload paper and add tags
\item
  The search function would return relevant result
\item
  The teacher can add students to add studies to the website's database
\item
  Students who are given the permission can upload paper and add tags
\item
  The user should be able to download the papers in the server.
\item
  The user should be able to see the thumbnail and successfully download
  the PDF
\item
  The storage should be large enough to store all the PDFs in the
  databse
\end{itemize}

\hypertarget{security}{%
\subsection{Security}\label{security}}

\begin{itemize}
\tightlist
\item
  Only teachers can permitted students can modify the database
\item
  Only the admins can modify the source code of the website
\end{itemize}

\hypertarget{accessibility}{%
\subsection{Accessibility}\label{accessibility}}

\begin{itemize}
\tightlist
\item
  The UI would still work when it's zoomed in or viewed on a smaller
  device like a mobile screen.
\item
  A screenreader should be able to read the content of the screen
  normally.
\end{itemize}

\end{document}
